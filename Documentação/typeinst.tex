%%%%%%%%%%%%%%%%%%%%%%%%%%%%%%%%%%%%%%%%%%%%%%%%%%%%%%%%%%%%%%%%%%%
%
% Ce gabarit peu servir autant les philosophes que les scientifiques ; 
% et même d'autres genres, vous en faites ce que vous voulez.
% J'ai modifié et partagé ce gabarit afin d'épargner à d'autres 
% d'interminables heures à modifier des gabarits d'articles anglais. 
% 
% L'ajout d'une table des matières et une bibliographie a été ajoutée,
% rendant le gabarit plus ajusté aux besoins de plusieurs.
%
% Pour retrouvé le gabarit original, veuillez télécharger les
% documents suivants: llncs2e.zip (.cls et autres) et 
% typeinst.zip (.tex). Les documents ci-haut mentionnés ne sont pas 
% disponibles au même endroit, alors je vous invite à fouiller le web. 
%
% Pour l'instant (02-2016) ils sont disponibles tous deux ici :
%
% http://kawahara.ca/springer-lncs-latex-template/
%
% Netkompt
%
%%%%%%%%%%%%%%%%%%%%%%%%%%%%%%%%%%%%%%%%%%%%%%%%%%%%%%%%%%%%%%%%%%%


%%%%%%%%%%%%%%%%%%%%%%% file typeinst.tex %%%%%%%%%%%%%%%%%%%%%%%%%
%
% This is the LaTeX source for the instructions to authors using
% the LaTeX document class 'llncs.cls' for contributions to
% the Lecture Notes in Computer Sciences series.
% http://www.springer.com/lncs       Springer Heidelberg 2006/05/04
%
% It may be used as a template for your own input - copy it
% to a new file with a new name and use it as the basis
% for your article.
%
% NB: the document class 'llncs' has its own and detailed documentation, see
% ftp://ftp.springer.de/data/pubftp/pub/tex/latex/llncs/latex2e/llncsdoc.pdf
%
%%%%%%%%%%%%%%%%%%%%%%%%%%%%%%%%%%%%%%%%%%%%%%%%%%%%%%%%%%%%%%%%%%%

\documentclass[runningheads,a4paper]{llncs}

\usepackage[utf8]{inputenc}

\usepackage{natbib}
\bibliographystyle{apalike-pt-br}

\usepackage{amssymb}
\setcounter{tocdepth}{3}
\usepackage{graphicx}

\usepackage[brazil]{babel} % Pour adopter les règles de typographie française

\usepackage[T1]{fontenc} % Pour que les lettres accentuées soient reconnues

\usepackage{url}
\urldef{\mailsa}\path|{alfred.hofmann, ursula.barth, ingrid.haas, frank.holzwarth,|
\urldef{\mailsb}\path|anna.kramer, leonie.kunz, christine.reiss, nicole.sator,|
\urldef{\mailsc}\path|erika.siebert-cole, peter.strasser, lncs}@springer.com|    
\newcommand{\keywords}[1]{\par\addvspace\baselineskip
\noindent\keywordname\enspace\ignorespaces#1}

\begin{document}

\mainmatter 

\title{Relatório Trabalho Prático II}

\titlerunning{Trabalho Prático II}

\author{Christoffer de Paula Oliveira, Paulo Henrique Tobias, Millas Násser}

\institute{UFSJ- Universidade Federal de São João Del Rei}

\authorrunning{Christoffer de Paula Oliveira, Paulo Henrique Tobias, Millas Násser}

\toctitle{Sumário}
\tocauthor{{}}

\maketitle

\medskip

\begingroup
\let\clearpage\relax
\tableofcontents
\addcontentsline{toc}{section}{Introduction}
\endgroup

\medskip
\medskip

\section*{Introdução}
    Este relatório tem como objetivo documentar as mudanças feitos no Trabalho Prático I para que este se adeque aos conceitos de Programação Orientada a Objetos (POO). Além disso, o relatório também relata o que ajudou e o que atrapalhou na implementação das novas funcionalidades requeridas no Trabalho Prático 2.

\section{Classe Jogo}
    A classe Jogo estava com um alto nível de acoplamento. Várias funções -- que, segundo a ideia de coesão, pertencem à outras classes -- estavam implementadas aqui. Além disso, havia muito código repetido, que mais tarde foi transformado em métodos. Estes fatores deixavam o código muito grande e quase ilegível.
    
    Praticamente todos os objetos do jogo eram declarados aqui. Além da lista de salas e trolls, também existiam listas para cada tipo de item. Estes itens, durante a inicialização, seriam espalhados aleatoriamente pelas salas. Para isso, existia uma classe \emph{Localizavel} que guardava a localização de todos os objetos do jogo (Incluindo Jogador e Trolls).

    A verificação e execução dos comandos do jogador também era feita nesta classe. Ações como pegar um item, por exemplo, que pertencem à classe Jogador, eram feitas na classe Jogo.

    \subsection{Mudanças e Justificativas}
        Para diminuir o acoplamento, foi criada uma nova classe chamada \emph{Console}, que faz o processamento da entrada do usuário e chama as funções corretas, cada uma implementada na sua devida classe. Portanto, se o usuário digitou o comando "pickup gold", o console agora apenas identifica que é o comando "pegar" e chama a função correspondente da classe Jogador.
        
        Além disso, uma nova classe \emph{Mapa} foi criada, que guarda a lista de salas e o objeto Jogador. A lista de trolls, assim como as de itens, foram transformadas em listas individuais para cada sala. Assim, cada sala passou a ter suas próprias listas de trolls e de itens.
        
        Ao final das modificações, a classe jogo apenas chama a função \emph{console} até que o jogo termine.

\section{Classe Sala}
    O principal problema das salas, além do já explicado acima, é que o método que inicializava o conjunto das salas estava implementado ali (veremos mais adiante que este foi um problema em quase todo o projeto). Podemos achar que esse tipo de método pertença à classe Sala, porém quem tem a função de inicializar as salas, ou seja, criar o mapa, é o próprio mapa.

    \subsection{Mudanças e Justificativas}
        A inicialização do conjunto de salas foi passada para a classe Mapa. Como dito anteriormente, as listas de trolls e itens foram movidas para a Sala. Por consequência disso, foram criados métodos para adicionar, remover, retornar e imprimir os itens e trolls das salas.

\section{Classe Jogador}
    Pouco coesa. Os métodos que deveriam estar aqui -- pegar, largar, mover, sair, atacar, etc -- estavam em outros módulos.
    
    Para armazenar o objeto (porta ou item) que estava perto do jogador eram usados dois atributos. Um do tipo \emph{Porta} -- para armazenar a porta -- e outro do tipo \emph{String} para armazenar os demais itens.

    \subsection{Mudanças e Justificativas}
        Largar e pegar itens, lançar machado, sair da sala e perder ouro, que são métodos responsáveis por realizar ações do jogador foram implementados aqui deixando o código mais coeso.
        
        Foi criada uma classe \emph{Aproximavel} que indica que o jogador pode se mover para perto deste objeto. Portanto, apenas um atributo é necessário para guardar esta informação. Com a remoção da classe \emph{Localizavel}, foi criado um atributo do tipo \emph{Sala} para armazenar a sala atual do jogador.

\section{Classe Localizável}
    Praticamente todo o código era dependente dessa classe.

    \subsection{Mudanças e Justificativas}
        Essa classe foi excluída ocasionando em um bom desnível de acoplamento no código,com isso todo o restante do código que necessitava dessa classe foi adaptado pra não depender mais. 

\section{Classes Chave, Machado, Pocao e Ouro}
    Todas possuíam o mesmo problema que a classe sala: possuíam métodos que aparentavam pertencer à classe mas que na verdade deveriam estar em outra.
    
    Índice de coesão extremamente baixo. Praticamente todos os métodos foram movidos para outros módulos.
    
    \subsection{Mudanças e Justificativas}
        Muitos métodos foram removidos e outros foram realocados. Os métodos de inicialização (presentes em todas as classes desta seção) foram transformados em um único método implementado na classe mapa.
        O efeito de cada item também passou a ser implementado dentro de cada classe, ou seja, se o jogador tenta usar uma chave, por exemplo, a função \emph{usar} da classe Chave é chamada e a porta é destrancada. %editei até aqui
        Para aumentar a coesão todos os itens passou a ter um método usar, estes métodos são responsáveis por aplicar ações de cada item no jogo, outro método criado é o de comparar, responsável por chamar a função compare da classe Util, que compara possíveis nomes que o usuário pode digitar com nomes semelhantes que representam os itens, consequentemente houve uma grande diminuição no código pois não foi mais utilizados switch/cases enormes. Houve também a criação classe Util para conter métodos que podem ser úteis em várias partes do código.

\section{Classe Porta}
    Não totalmente coesa, desconsiderando set e get, havia apenas um método,de pegar identificador a partir de um array de portas, sendo que o mesmo não deveria estar nessa classe.

    \subsection{Mudanças e Justificativas}
        Para aumentar a coesão foi criado um método para para chamar a compare da classe util, funcionando de maneira idêntica ao método comparar em itens, ou seja, a partir do comando digitado pelo usuário. 

\section{Classe Troll}
    Acoplada por depender de sala e também com baixa coesão, métodos de outras classes, como inicializa troll, estavam sendo implementados nessa classe.

    \subsection{Mudanças e Justificativas}
        Métodos de outras classes foram movidos para os seus devidos lugares, e método de mover trolls modificado para se adaptar ao novo código.

\section{Classe TrollNome}
    Uma função de troll que poderia estar implementada na classe Troll. 

    \subsection{Mudanças e Justificativas}
        Caso precise ser mais detalhado

\section{Pontos positivos}

Houve também pontos em que o código estava seguindo a Programação Orientada a Objeto. O encapsulamento foi feito de maneira correta, As classes Mochila e Pegavel para o antigo código estava sem problemas,isto é, nenhum erro de coesão ou acoplamento 

\section{Mudanças gerais}

	A partir do momento que erros de Programação Orientada a Objeto foi corrigidos, houve mudanças para a realização da expansão do código. O jogo passou a ter outras características tal como especificado na documentação da tarefa.

	\subsection{Exceções}
		O código inicial não se provia de nenhum meio para a detecção de exceções lançadas pelo jogo. Portanto os poucos meios de se evitar algum erro eram feitos a partir de retornos de funções que por sua vez tornavam o código um pouco dificil de se ler já que não havia nenhuma especificação sobre.

	\subsubsection{Mudanças e Justificativas}
		Para tal foi criado três conjuntos de exceções
		\begin{itemize}
			\item Personagem
			\item Item
			\item Aproximavel
		\end{itemize}

		Exceções do tipo personagem, são responsáveis por lançar mensagens para trolls, que são por sua vez indicam quais movimentos ele poder fazer e sobre o que ele pode interagir. Já para o jogador, estas fazem parte de todas as ações do mesmo, partindo desde interação de algum objeto presente, movimentação, e se o mesmo escreveu algum comando errado.
		
		Exceções geradas por algum item geralmente é que os mesmos não puderam realizar suas devidas ações, como exemplo abrir portas ou usar poção em algum lugar inválido.
		
		Por fim a última excessão a ser gerada pelo programa acontece pela interação das duas anteriores, estas geram um erro quando o personagem tenta se aproximar de algum item em específico e não foi possivel realizar a ação desejada como chegar perto de portas que não existem na sala e entre outros.

\section{Evolução}
	Para essa nova abordagem, o jogo apresenta novas funcionalidades.
	
	\subsection{Salas}
		Salas agora possuem tamanhos diferentes, com isso as salas serão divididos em metros quadrados que por sua vez visam limitar a quantidade de ouro disponível, onde no máximo podem caber dez peças de ouro por metro quadrado. Portanto salas de tamanhos diferentes armazenam quantidades diferentes.
		
		Com essa modificação no tamanho, chegamos a outro ponto, elas por sua vez não podem possuir a mesma configuração de tamanho.
		
		Para isso foi adicionado um novo campo na classe Sala, que é tamanho de cada uma para saber identificar qual a quantidade certa de ouro. E a criação das mesmas agora são feitos a partir de um arquivo json, diferentemente de antes que era feita a partir da sua própria classe. Deixando o código mais limpo.
		
		%Continuar sala
	
	\subsection{Machados}
		Para essa nova abordagem, o jogo apresenta novas configurações de machados.
		\begin{itemize}
			\item Machado de ouro
			\item Machado de bronze
			\item Machado de ferro
		\end{itemize}
		
		Para isso a classe Machado anteriormente se tornou abstrata e recebeu dois novos atributos. 
		
		O primeiro deles foi a durabilidade, essa indica qual a quantidade de vezes o machado será lançado e o outro campo é o material, fornecendo a durabilidade individual de cama machado.
		
		Machados de ferro possuem durabilidade um, ou seja, posem ser lançados apenas uma vez. Os de bronze podem ser lançados em até duas vezes e por fim machados de ouro são lançados 5 vezes.
		
		A abordagem foi criar mais 3 novas classes, uma para cada material acima descrito. Todas elas possuem funcionamento em comum, portanto todas herdaram da classe machado.
		
		Em seu lançamento cada machado é responsável por saber quem é seu alvo de ataque e qual seu efeito sobre ele, o se lançar em um Troll este será diminuído em sua durabilidade. Mas se o ataque for ao jogador então será mandado um sinal para a classe mochila que ficará a cargo de remover os itens necessários, dependendo da durabilidade do machado.
		
	\subsection{Jogador}
		A principal mudança realizada foi expandir a mochila e limitar a quantidade de itens de cada tipo que o jogador pode carregar. Pode-se agora levar até quatro machados, independente de seu material, poções e chaves são possíveis carregar até três delas, não mais que isso. 
		
		Para simplificar o processo a mochila agora conta com três listas de cada item descrito acima. E ao fazer a verificação é necessário apenas ver o tamanho de cada uma. 

\section{Conclusão}

{\small (Example from Jensen K., Wirth N. (1991) Pascal user manual and
report. Springer, New York)}

\bibliography{references}
\nocite{*} 

\end{document}

