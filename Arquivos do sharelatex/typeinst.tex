%%%%%%%%%%%%%%%%%%%%%%%%%%%%%%%%%%%%%%%%%%%%%%%%%%%%%%%%%%%%%%%%%%%
%
% Ce gabarit peu servir autant les philosophes que les scientifiques ; 
% et même d'autres genres, vous en faites ce que vous voulez.
% J'ai modifié et partagé ce gabarit afin d'épargner à d'autres 
% d'interminables heures à modifier des gabarits d'articles anglais. 
% 
% L'ajout d'une table des matières et une bibliographie a été ajoutée,
% rendant le gabarit plus ajusté aux besoins de plusieurs.
%
% Pour retrouvé le gabarit original, veuillez télécharger les
% documents suivants: llncs2e.zip (.cls et autres) et 
% typeinst.zip (.tex). Les documents ci-haut mentionnés ne sont pas 
% disponibles au même endroit, alors je vous invite à fouiller le web. 
%
% Pour l'instant (02-2016) ils sont disponibles tous deux ici :
%
% http://kawahara.ca/springer-lncs-latex-template/
%
% Netkompt
%
%%%%%%%%%%%%%%%%%%%%%%%%%%%%%%%%%%%%%%%%%%%%%%%%%%%%%%%%%%%%%%%%%%%


%%%%%%%%%%%%%%%%%%%%%%% file typeinst.tex %%%%%%%%%%%%%%%%%%%%%%%%%
%
% This is the LaTeX source for the instructions to authors using
% the LaTeX document class 'llncs.cls' for contributions to
% the Lecture Notes in Computer Sciences series.
% http://www.springer.com/lncs       Springer Heidelberg 2006/05/04
%
% It may be used as a template for your own input - copy it
% to a new file with a new name and use it as the basis
% for your article.
%
% NB: the document class 'llncs' has its own and detailed documentation, see
% ftp://ftp.springer.de/data/pubftp/pub/tex/latex/llncs/latex2e/llncsdoc.pdf
%
%%%%%%%%%%%%%%%%%%%%%%%%%%%%%%%%%%%%%%%%%%%%%%%%%%%%%%%%%%%%%%%%%%%

\documentclass[runningheads,a4paper]{llncs}

\usepackage[utf8]{inputenc}

\usepackage{natbib}
\bibliographystyle{apalike-pt-br}

\usepackage{amssymb}
\setcounter{tocdepth}{3}
\usepackage{graphicx}

\usepackage[brazil]{babel} % Pour adopter les règles de typographie française
\usepackage[T1]{fontenc} % Pour que les lettres accentuées soient reconnues

\usepackage{url}
\urldef{\mailsa}\path|{alfred.hofmann, ursula.barth, ingrid.haas, frank.holzwarth,|
\urldef{\mailsb}\path|anna.kramer, leonie.kunz, christine.reiss, nicole.sator,|
\urldef{\mailsc}\path|erika.siebert-cole, peter.strasser, lncs}@springer.com|    
\newcommand{\keywords}[1]{\par\addvspace\baselineskip
\noindent\keywordname\enspace\ignorespaces#1}

\begin{document}

\mainmatter 

\title{Relatório Trabalho Prático II}

\titlerunning{Trabalho Prático II}

\author{Christoffer de Paula Oliveira, Paulo , Millas}

\institute{UFSJ- Universidade Federal de São João Del Rei}

\authorrunning{Christoffer de Paula Oliveira, Millas Násser, Paulo Henrique Tobias}

\toctitle{Sumário}
\tocauthor{{}}

\maketitle

\medskip

\begingroup
\let\clearpage\relax
\tableofcontents
\addcontentsline{toc}{section}{Introduction}
\endgroup

\medskip
\medskip

\section*{Introdução}

O objetivo deste relatório é citar e justificar mudanças realizadas no Trabalho Prático I do aluno Renan.As mudanças realizadas visam um melhoramento no código java, com relação a Programação Orientada a Objeto e também implementação de novas funcionalidades.

\section{Classe Jogo-Main}

A classe Jogo a princípio estava com um alto nível de acoplamento, além de deixar o código extremamente extenso, com pouco necessidade, haviam também implementação de métodos, que deveriam estar em outras classes, como por exemplo: Inicialização de objetos, ações do jogador pegar/largar, utilização de poções etc.

\subsection{Mudanças e Justificativas}

Para resolver parte do acoplamento o método console foi simplificado e transformado em uma classe, essa classe ficou responsável por tratar os comandos e transforma-los em ações no jogo chamando métodos, se necessário, em outras classes, criou-se também a classe Mapa que ficou responsável por armazenar inicializar e juntar todos os objetos e o personagem do jogo, para remover o restante do acoplamento os de mais métodos foram movidos para suas respectivas classes com alterações para adaptar-se a nova estrutura do código.

\section{Classe Sala}

Não tinha objetos como trolls e itens apenas portas. Havia problemas de acoplamento, sendo feita a inicialização das salas, colocando portas, e também obtendo nome a partir de um array de salas, que ficava sendo passada para os métodos na classe. Pouco coesa, pois métodos para exibir e retornar itens disponíveis na sala estava em outras classes.

\subsection{Mudanças e Justificativas}

Para a cada Sala ter todos os objetos que se pode ter em uma sala foi declarado um array de Pegavel e Trolls, toda a inicialização como dito anteriormente foi passada para a classe Mapa, criados métodos para retornar, remover, adicionar e imprimir itens e trolls nas salas.

\section{Classe Jogador}

Pouco coesa. Nenhum método além de get e set implementado nessa classe ações do jogador como, soltar e pegar, estavam implementadas na classe Jogo. Acoplada por haver tipo String para representação de um item.

\subsection{Mudanças e Justificativas}

Métodos que são responsáveis por realizar ações do jogador foram implementados aqui, ações como largar e pegar itens, lançar machado, sair da sala e perder ouro deixando o código mais coeso. O tipo de variável para representar o objeto que está perto passou a ser Aproximavel, falaremos dessa classe mais adiante. Por ultimo foi criada uma variável do tipo Sala para armazenar onde o jogador se encontra atualmente no mapa. 

\section{Classe Localizável}

Muito acoplada, praticamente todo o código era dependente dessa classe.

\subsection{Mudanças e Justificativas}

Essa classe foi excluída ocasionando em um bom desnível de acoplamento no código,com isso todo o restante do código que necessitava dessa classe foi adaptado pra não depender mais. 

\section{Classes Chave, Machado, Pocao e Ouro}

Todas estão na mesma forma, ou seja implementação bem semelhante, por isso ficou desnecessário criar tópicos diferentes.
Índice de coesão extremamente baixo, praticamente todos os métodos não eram coesos, por exemplo imprimir itens na sala sem as classes sequer terem salas. 

\subsection{Mudanças e Justificativas}

Para aumentar a coesão todos os itens passou a ter um método usar, estes métodos são responsáveis por aplicar ações de cada item no jogo, outro método criado é o de comparar, responsável por chamar a função compare da classe Util, que compara possíveis nomes que o usuário pode digitar com nomes semelhantes que representam os itens, consequentemente houve uma grande diminuição no código pois não foi mais utilizados switch/cases enormes. Houve também a criação classe Util para conter métodos que podem ser úteis em várias partes do código.

\section{Classe Porta}

Não totalmente coesa, desconsiderando set e get, havia apenas um método,de pegar identificador a partir de um array de portas, sendo que o mesmo não deveria estar nessa classe.

\subsection{Mudanças e Justificativas}

Para aumentar a coesão foi criado um método para para chamar a compare da classe util, funcionando de maneira idêntica ao método comparar em itens, ou seja, a partir do comando digitado pelo usuário. 

\section{Classe Troll}

Acoplada por depender de sala e também com baixa coesão, métodos de outras classes, como inicializa troll, estavam sendo implementados nessa classe.

\subsection{Mudanças e Justificativas}

Métodos de outras classes foram movidos para os seus devidos lugares, e método de mover trolls modificado para se adaptar ao novo código.

\section{Classe TrollNome}

Uma função de troll que poderia estar implementada na classe Troll. 

\subsection{Mudanças e Justificativas}


Caso precise ser mais detalhado

\section{Pontos positivos}

Houve também pontos em que o código estava seguindo a Programação Orientada a Objeto. O encapsulamento foi feito de maneira correta, As classes Mochila e Pegavel para o antigo código estava sem problemas,isto é, nenhum erro de coesão ou acoplamento 

Complexidade do algoritmo aqui




\paragraph{Remark 1.}
Paragrafo


\section{analise de Resultados}
Analise aqui

\section{Conclusão}

{\small (Example from Jensen K., Wirth N. (1991) Pascal user manual and
report. Springer, New York)}

\bibliography{references}
\nocite{*} 

\end{document}
